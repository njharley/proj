% Created 2014-03-27 Thu 15:17
\documentclass{article}
\usepackage[utf8]{inputenc}
\usepackage[T1]{fontenc}
\usepackage{fixltx2e}
\usepackage{graphicx}
\usepackage{longtable}
\usepackage{float}
\usepackage{wrapfig}
\usepackage{soul}
\usepackage{textcomp}
\usepackage{marvosym}
\usepackage{wasysym}
\usepackage{latexsym}
\usepackage{amssymb}
\usepackage{hyperref}
\tolerance=1000
\usepackage{natbib}
\providecommand{\alert}[1]{\textbf{#1}}

\title{Literature Review}
\author{Nicholas Harley}
\date{\today}
\hypersetup{
  pdfkeywords={},
  pdfsubject={},
  pdfcreator={Emacs Org-mode version 7.9.3f}}

\begin{document}

\maketitle

\setcounter{tocdepth}{3}
\tableofcontents
\vspace*{1cm}

\section{Introduction}
\label{sec-1}

\begin{itemize}
\item objectives
\item description of tonality
\item similarity
\item visualisation
\item introduce musicological/perceptual ingrediets to MIR
\end{itemize}
\section{Tonality}
\label{sec-2}
\subsection{Defining Tonality}
\label{sec-2-1}

Tonality is a notoriously complex musical phenomenon and numerous
definitions have been proposed from a variety of viewpoints. Perhaps
the most general definition is that provided by Hyer2013: ``\ldots{} refers
to the systematic arrangements of pitch phenomena and relations
between them''. Explanations of tonality have been provided through
many different disciplines (acoustics, music theory, linguistics,
cognitive psychology) and a detailed discussion of these areas is
certainly beyond the scope of this work. However, it is generally
agreed that tonality is an abstract cultural and cognitive contruct
that can have many different physical representations.

\citep{Babbitt1965} proposed three domains to categorise different
types of representation of music: acoustic (physical), auditory
(percieved), graphemic (notated). Western music theory provides a
lexicon for describing abstract tonal objects with terms such as note,
chord and key. These objects have a hierarchical relationship and the
meaning of these labels is highly dependent on musical context and the
scale of obsevation. Musicological descriptions, which constitute the
majority of reasoning about tonailty, reside mainly in the Babbitt's
graphemic domain, although arguably they refelct some aspects of the
other two. Each domain, whilst connected to every other, provides only
a projection of the musical whole and definitions of tonality that
confine themselves to just one will certainly be incomplete. However,
they provide a convenient framework for the discussion that follows.
\subsection{Modelling Tonality}
\label{sec-2-2}

The challenge of mathematically modelling aspects of tonality has been
approached in numerous ways and from differnt domains. In the
graphemic domain, musicologists and composers have proposed
theoretical models, attempting to rethink tonal theory from a
mathematical perspective. These models employ different branches of
mathematics such as geometry (\citep{Tymoczko2012}) or group theory
(Steven Ring) to describe harmonic structure. From the auditory
domain, cognitive psychologists have built models on tonal induction
based on preceptual ratings of tonal stimuli (\citep{Krumhansl1990}).
\subsubsection{Tonality as Context}
\label{sec-2-2-1}

Many models approach the concept of tonality as a context, within
which the relations and hierarchies of tonal phenomena can be
understood. A sense of tonality can be induced when musical stimuli
resembles some a priori contextual category. For western music of the
major-minor period, key signatures comprise a collection of categories
that give context to the tonal components of
music. \citep{Martorel2013} identifies three important aspects of
tonality as context: dimensionality (the relatedness of categories),
ambiguity (reference to two or more categories simultaneously) and
timing (the dynamics of tonal context). Adequate models of tonality
should be capable of quantifying these aspects.
\subsubsection{Tonality in MIR}
\label{sec-2-2-2}

The MIR community is primarily concerned with the extraction of tonal
descriptors from audio signals such as chord and key estimates. Most
systems use chroma features as a prelimiary step, obtained by mapping
STFT or constant Q transform (REF) energies to chroma bins. Template
matching is used to compare the chroma vectors to a tonal model using
some distance measure. A commonly used tonal model for key estimation
are the KK-profiles (\citep{Krumhansl1990}) (used by ). Distance
measures such as inner product (eg in ) and fuzzy distance (eg in )
are used to compare vectors. Statistical methods such as HMMs have
been used for chord and key tracking tracking (REF). Of addition
interest in the field is the concept of musical similarity (for music
recommendation, structure analysis and cover detection, for
example). Foote1999 computed self-similarity matrices for
visualisation of structure by correlating the MFCC feature vector
time-series. Gomez2006 proposed the application of this method to
tonal feature vectors.
\subsubsection{Similarity}
\label{sec-2-2-3}

The importance of defining the similarity or closeness between musical
phenomena, be it theoretical, physical or perceptual, is central to
almost every model of tonality and often leads to a geometric
configuration tonal objects. The concept of similarity and distance is
discussed further in Chapter \ref{sec-5} where a review
of spatial models of tonality is given.
\subsection{The Semantic Gap}
\label{sec-2-3}
\subsubsection{Acoustic Domain}
\label{sec-2-3-1}

\citep{Wiggins2009} discusses, what is refered to in MIR as, the
``Semantic Gap'': the inability of systems to achieve success rates
beyond a conspicous boundary. He examines the fundamental
methodological groundings of MIR in terms of Babbitts three domains,
discussing the limits of each representation and regarding the
discarnate nature of music. He concludes that the audio signal
(acoustic domain) simply connot contain all of the information that
systems seek to retrieve. He points towards the the auditory domain as
the chief residence of music information and urges for in not to be
overlooked in MIR and wider music research.
\subsubsection{Graphemic Domain}
\label{sec-2-3-2}

Furthermore he criticises the purely graphemic approach and the
tendency of music research to presuppose musicological
axioms. \citep{Wiggins2012} argues that music (tonal) theory is,
rather than a theory in the scientific sense, a highly developed folk
psychology (internal human theory for explaining common
behaviour). Thus, the rules of music theory are not like scientific
laws but rather abstract descriptions of a specific musical
behavior. This idea challenges the validity of formalizing such rules
in mathematics and promts the question, ``What is actually being
modelled?'' He concludes that to apply mathematical models to muscial
output alone (scales or chords) without consideration of the musical
mind is a scientific failure.
\subsubsection{Problems}
\label{sec-2-3-3}

The two assersions of Wiggins sit contrary to a number of the aspects
of the tonal models discussed in Chapter \ref{sec-2-2}. Firstly,
the major-minor paradigm, upon which so many approaches are based,
whilst certainly possessing cognitive significance, is still a
musicological concept and therefore a misleading basis for both
mathematical and cognitive approaches. A second problem is that of the
numerical methods used by some MIR systems, in particular, distance
measures. As will be discussed in Chapter XXXX, similary (and by
extension distance) is a central part of the auditory domain. MIR
systems often uses distance measures from mathematics such as
Mahalanobis (Tzanetakis99) or Cosine (Foote00) with little
consideration of their perceptual or musical significance.
\subsection{Systemasticity}
\label{sec-2-4}

Having cautioned against a purely musicological approach,
\citep{Wiggins2009} proposes a compromise: to adopt a bottom-up
approach to music theory, exploring the concepts through systematic
mid-level representations. He states that ``\ldots{}methods starting at, for
example, the musical surface of notes is a useful way of
proceeding\ldots{}'' The concept of musical surface is illustrated by
\citet[pp. 159]{Huovinen2007} with a metaphore: ``\ldots{}to approach a
musical landscape not by drawing a map, which necessarily confines
itself to a limited set of structurally important features, but by
presenting a bird’s-eye view of the musical surface – an aerial
photograph, as it were, which details the position of every pitched
component.'' 

\citep{Martorell2013} also advocates this approach, observing that
surface description influences analyitcal observation and, for an
unbiased view, the analyst must be provided with the adequate raw
materials with which to make more indepth observation. Such a
systematic, descriptive model would be fundamentally independent of
high level concepts such as chords and key but, at the same time,
capable of capturing them. \citep{Martorell2013} also disusses the
importance of systemasticity in terms of the dimensionality, ambiguity
and timing. He finds that models based on the major-minor paradigm are
incapable of adequately describing tonal ambiguity even in some
westerm music (\citep[chap. 3]{Martorell2013}).

With a systematic description of the musical surface, theories and
models from different domains can be gathered and evaluted together in
the same analytical arena, thus helping to bridge the gap between
traditional musicology, cognitive psycology and MIR.
\section{Pitch Class Set Theory}
\label{sec-3}

One such method availble for systematic description of the musical
surface is Pitch class set theory. PC-set theory is a system for
analysing the pitch content of music. It uses class equivelance
relations to reduce the ammount of data required to describe any
sequence of pitches.
\subsection{Pitch Class}
\label{sec-3-1}

Pitch-class set theory uses octave equivelance. A Pitch-Class (PC) is
a number indicating the position of a note within the octave. For
example, in western equal temperement (TET) the octave is divided in
to 12 steps. Each of these notes is given a number from 0-11.
\begin{table}[htb]
\caption{Pitch classes} 
\begin{center}
\begin{tabular}{lrrrrrrrrrrrr}
 Note  &  C  &  C\#  &  D  &  D\#  &  E  &  F  &  F\#  &  G  &  G\#  &  A  &  A\#  &   B  \\
 PC    &  0  &    1  &  2  &    3  &  4  &  5  &    6  &  7  &    8  &  9  &   10  &  11  \\
\end{tabular}
\end{center}
\end{table}

A PC-set is a collection of PCs ignoring any repetitions and the order
in which they occur. PC-sets are notated as follows \{0,1,2,3,4\} with
PCs ordered from lowest to highest as a convention.

Example: Pitch-set S = \{A4,C5,E5,A5\} (A minor chord) so PC-set S =
\{9,0,4,9\} = \{0,4,9\}

The cardinality of a set, denoted \#S, is the number of PCs it
contains. There are 4096 (2$^{\mathrm{12}}$) unique PC-sets and any segment of
music can be represented as a PC-set. Ignoring the octave, the order
of the notes and whether they occur simultaneously or in sequence
means PC-sets are abstract objects with many different musical
manifestations.
\subsection{Set Classification}
\label{sec-3-2}

Defining equivelance classes of PC-sets further reduces the total
number of tonal objects. A Set Class (SC) is a group of PC-sets
related by a transformation or group of transformations. The two types
of transformation commonly used are transposition and inversion. A
transposition, Tn(S), transposes the set S by the interval n (by
adding n to all PCs). An inversion, I(S), inverts the set S meaning
all PCs are replaced by their inverse (11-PC). 

Example: S = \{0,4,9\}, T3(S) = \{3,7,0\} = \{0,3,7\}

Example: S = \{0,4,9\}, I(S) = \{11,7,2\} = \{2,7,11\}

From these two transformations it is possible to define three types of
SC.

\begin{table}[htb]
\caption{SC types} 
\begin{center}
\begin{tabular}{ll}
 Transpositional (Tn)  &  All PC-sets that can be transformed to each by  \\
                       &  transposition belong to the same class.         \\
                       &  There are 348 distinct Tn types.                \\
 Inversional (I)       &  All PC-sets that can be transformed to each     \\
                       &  otherby inversion belong to the same SC.        \\
                       &  There are 197 distinct I types. All PC-sets     \\
 Transpositional/      &  that can be transformed to each other by        \\
 Inversional (Tn/I)    &  transposition, inversion or both belong to the  \\
                       &  same SC. There are 220 distinct Tn/I types.     \\
\end{tabular}
\end{center}
\end{table}


The Prime Form of a PC-set is a convention for denoting the SC it
belongs to. The convention was introduced by Allan Forte for Tn/I
types and has since been adopted by the majority of theorists. In
addition he devised a system for ordering Tn/I SCs and assigning to
each one a number. For example, the Forte number 3-11 refers to the
11th SC of cardinality 3. This convension has been modified for use
with Tn types by adding A and B to the names of inversionally related
SCs.

One additional concept is that of cardinality class (nC) which refers
to all the SCs of cardinality n. Cardinality class 2 is commonly
refered to as Interval class (IC). There are 6 distinct intervall
classes.

\begin{table}[htb]
\caption{Forte's Prime form and numbering convention} 
\begin{center}
\begin{tabular}{ll}
 PC-set             &  \{1,4,9\}  \\
 Prime Form (Tn/I)  &  \{0,3,7\}  \\
 Prime Form (Tn)    &  \{0,4,7\}  \\
 Forte Name (Tn/I)  &  3-11       \\
 Forte Name (Tn)    &  3-11B      \\
\end{tabular}
\end{center}
\end{table}


\begin{table}[htb]
\caption{SC type} 
\begin{center}
\begin{tabular}{lr}
 Object type  &  No. Objects  \\
\hline
 Pitch        &           88  \\
 Pitch set    &         3e26  \\
 PC           &           12  \\
 PC-set       &         4096  \\
 Tn-Type      &          348  \\
 I-Type       &          197  \\
 Tn/I-Type    &          200  \\
\end{tabular}
\end{center}
\end{table}
\subsection{Vector Analysis}
\label{sec-3-3}
\subsubsection{Membership and Inclusion}
\label{sec-3-3-1}

Two concepts that are crucial in PC-set theory are Membership and
Inclusion. Membership of a set is denoted q $\in$ S and means that PC p
is a member of set S. Inclusion in a set is denoted Q $\subset$ S and
means that all members of Q are also members of set S. Q is said to be
a subset of S.

Example: 4 $\in$ \{0,4,9\}

Example: \{0,4,9\} $\subset$ \{0,1,4,5,9\}
\subsubsection{Embedding Number}
\label{sec-3-3-2}

Lewin \citep{Lewin1979} applied these concepts to SCs to develop his
Embedding Number, EMB(X,Y). Given two SCs, X and Y, EMB(X,Y) is the
number of instances of SC X which are included in (are subsets of) SC
Y.

Example: X = \{0,4\} and Y = \{0,4,8\} so EMB(X,Y) = 3
\subsubsection{Subset Vectors}
\label{sec-3-3-3}

An n-class subset vector of X, nCV(X), is therefore an array of values
of EMB(A,X) where A is each of the SCs in the cardinality class
nC. The Interval-Class Vector (ICV) is a special instance of the nCV
with n equal to 2.

Example: X = \{0,4,9\}, 2CV(X) = ICV(X) = [0 0 1 1 1 0]

Subset vectors forms the basis of the majority of analysis performed
by PC-set theorists. In addition, many theorists have proposed
modifications to the basic nCV to suit their specific purposes and
some of these modifications will be discussed in context where
necessary.
\subsection{PC-Set Similarity}
\label{sec-3-4}
\subsubsection{Similarity Relations}
\label{sec-3-4-1}

The assessment of similarity between SCs has been discussed in the
literature for decades and a large number theoretical models have been
proposed. Different models approaches the problem from different
conceptual standpoints and theorists have different opinions about the
contributing factors. All these models are described under the blanket
term ``similarity relations''. Despite the perennial fascination with
the concept, little or no concensus exits as to what constitutes a
good similarity relation.

Extensive analysis and comparison of many similarity relations has
been done by \citep{Isaacson1992}, \citep{Buchler1997},
\citep{Hermann1994} and \citep{Castren1994}. In addition 

Castren provides a comprehensive and indepth review of a large number
of similarity relations and categorises them according to some
fundamental principles.

Plain Relations: Plain relations, such as Forte's R-relations
(citep\{Forte1973\}), do not produce a range of values indicating
similarity. Instead they indicate whether the the SCs are related in a
specific way which in turn may give some indication of whether they
are similar.

Single Subset nC: Single subset nC relations consider subsets of a
particular cardinality that are contained in two SCs. This done by
comparing the nCVs of the two SCs for a certain value of n. Many of
the relations in this category compre the ICVs (2CVs). The measures of
\citep{Morris1979}, \citep{Teitelbaum1965}, \citep{Lord1981},
\citep{Isaacson1990}, \citep{Rahn1979}, \citep{Rogers1999} are all
included in this category.

Total measure: Total measures consider the subsets of all
cardinalities contained within in two SCs. All the relevant nCVs are
compared to produce a final value. Total measures have been proposed
by \citep{Rahn1979}, \citep{Lewin1979}, \citep{Castren1994} and
\citep{Buchler1997}.

In addtion to his categorisation, \citep{Castren1994} proposes several
criteria which a good similarity relation should meet.

\begin{itemize}
\item criteria
\begin{itemize}
\item Isaacson
\item Buchler
\item Hermann
\item Castren
\item vast number of measure
\item how to limit the scope
\item won't consider: set complexes, permutation families, regions,
    harmonic genera, ordering of sets
\item context free, non transitive
\item only the with SC types based on Tn/TnI
\item vector based
\end{itemize}
\item all these similarity measure are pure quantitive not qualitative. is
  there perceptual relevance?
\end{itemize}
\subsubsection{Similarity Measures}
\label{sec-3-4-2}
\subsection{Perceptual relevance}
\label{sec-3-5}
\subsubsection{Octave equivelance}
\label{sec-3-5-1}

\begin{itemize}
\item Pitch is a percept that derives from a particular harmonic structure
  and is roughly proportional to the logarithm of the fundamental
  frequency. This allows pitch to be modelled as a straight line,
  finite at both ends.
\item Music psychologists have observed a strong perceptual similarity
  between pitches with fundamental frequencies in the ratio of 2:1
  (REF). This property of octave similarity leads the straight line
  model of pitch to be bent into a helix.
\item Division of the octave into a number of categories is thought to
  offer a more effecient cognitive representation in memory and thus
  confers evolutionary advantage. The resulting pitch equivelance
  classes are implicitly learned through exposure at an early age. TET
  has 12 pitch equivelance classes which in PC-set theory are modelled
  as a circular projection of the pitch helix.
\item Thus the fundamental compent of PC-set theory would appear to have a
  solid basis in percepton.
\item Bruner1988 investigated the percieved similarity of pairs of chords
  with varying numbers of octave related pitches. He found that in
  general chords with identical PC contents were percieved as more
  similar than chords with near identicle PC contents.
\item Bruner1993 carried out further experiments using hexchords. In these
  experiemts shared PC contents of two chords did not appear to
  influence the subjects' similarity ratings, a result that would
  appear to contradict his earlier study.
\end{itemize}
\subsubsection{Set class equivelance}
\label{sec-3-5-2}

\begin{itemize}
\item Some researchers have attempted to examine whether there is
  percieved equivelance between different manifestations of a
  PC-set. Krumhansl et al, 1987 presented subjects with sequences of
  tones derived by transforming two different PC-sets. They noted that
  subjects were able to distinguish between the different sets both in
  neutral and musical contexts.
\item Millar1984 investigated the peceptual similarity of different
  PC-sets derived from the same set class under Tn/I
  classification. Subjects were presented with 3 note melodies and
  asked to judge which was equivelant to a reference melody. Some
  melodies preserved the SC identity whilst others did not. He found
  transpositions to be percieved more similar than inversions and in
  addition he descovered that the order of the notes and
  melodic contour was a strong factor in perceived similarity.
\item Some authors have questioned the perceptual relevance of using Tn/I
  equivelance as a basis for set classification. Deutsch, 1982 seems
  unconvinced by evidence for the perceptual similarity of inverted
  intervals. Cross, 1985 is also sceptical and uses the example of
  major and minor triads which, while perceptually distinct, are
  equivelant under Tn/I.
\end{itemize}
\subsubsection{Percieved vs Theoretical Similarity}
\label{sec-3-5-3}

\begin{itemize}
\item A number of studies have been done on the connection between
  perceptual similarity ratings and the theoretical values obtained
  from some PC-set similarity measures.
\item Bruner1984 used multidimensional scaling on subjects' similarity
  ratings between trichords and tetrachords and on the similarity
  values obtained from SIM (\citep{Morris1979}). She compared the
  2-dimensional solutions and found there to be little
  correlation.
\item Gibson1986 investigated notraditional chords. He compared subjects'
  ratings with similarity assessments calculated from Forte's
  R-relations (\citep{Forte1973}) and Lord's similarity function
  (\citep{Lord1981}). He also concluded there was little
  correspondence between the two.
\item Stammers1994 compared subjects' ratings of 4 note melodes with the
  theoretical values obtained from SIM (\citep{Morris1979}). She found
  the ratings of subjects with more musical training to be more
  correlated with the SIM values.
\item Lane1997 compared subjects' ratings of pitch sequences with
  corresponding values of seven ICV-based similarity measures: ASIM,
  MEMB2, REL2, s.i., IcVSIM and AMEMB2 concluded there to be a strong
  relation.
\item citep\{Kuusi2001\} investigated the connection between theoretical
  similarity and percieved similarity. Subjects were asked to rate the
  closeness of pentachords and their ratings were compared to the
  theoretical values obtained from 9 theoretical similarity
  measures. He found there to be a connection between aurally
  estimated ratings and the theoritcal values and concluded that the
  abstract properties of set-classes do have some perceptual
  relevance.
\end{itemize}
\subsection{PC set theory for Analysis}
\label{sec-3-6}

\begin{itemize}
\item Little has been done
\item Isaacson
\item Huovinen, Tenkanen
\item Martorell timescale
\end{itemize}
\section{Multi Dimensional Scaling}
\label{sec-4}

\begin{itemize}
\item overview
\item different techniques
\item dimensionality
\item stress
\item goodness of fit
\item Samplaski etc
\end{itemize}
\section{Spatial Models of Tonality}
\label{sec-5}
\subsection{Similarity and Distance}
\label{sec-5-1}

Judgements of similarity form the basis of many cognitive processes
including the perception of tonality. Similarity between two objects
is often concieved as being inversly related to distance between them
in geometric space. For example, some tonal objects (chords, for
example) are percieved as close to one another whereas others are
further apart. In addition, the number of dimensions of the geometric
space is in connection with the number of independent properties that
are relevant for similarity comparisons.\citep{Gardenfors1995}
suggests that humans are naturally predisposed to create spatial
cognitive representations of percetual stimulai due to the geometric
nature of the world we have evolved to inhabit. Therefore spatial
modelling of tonalilty, as well as helping to visualise the complex
multidimenstional relationships between tonal phenomena, has the
potential to refelct cognitive aspects of the way they are percieved.
\subsection{Spatial Representations}
\label{sec-5-2}

Throughout history theorists have proposed many spatial
representations of tonality from different domains. From the graphemic
domain, \citep{Heinichen1728}, \citep{Webber1817} and
\citep{Schoenberg1948} all proposed simple 2-dimensional charts to
display the proximity between keys. For representation of chords,
\citep{Riemann} models major and minor triads as regions in a
2-dimensional space whilst \citep{Tymoczko2011} propeses a vaireiety
high dimensional, non-euclidean chord spaces that reflect the
theoretical principles of voice leading. From the acoustic domain,
\citep{Shepard1982} proposes a five-dimensional model to represent
interval relations between pitches. Some theorists have attepted
encorporate relations between several levels of tonal hierarchy into
one configuration. The ``spiral array'' of \citep{Chew2000} is a
three-dimensional mathematcal model which simulataneously captures the
relations between pitches, chords and keys. The chordal-regional space
of \citep{Lerdahl2001} models the relations between chords within a
certain key.
\subsection{Cognitive Psychology}
\label{sec-5-3}

The auditory domain has been addressed through congnitive psychology
by \citep{Krumhansl1990} who used the probe-tone methodology
(\citep{Krumhansl1979}) to establish major and minor key profiles (12
dimensional vectors containing the perceptual stability ratings of
each of the 12 pitch classes within a major or minor context). These
profiles, know as Krumhansl-Kessler profiles (KK-profiles), show the
hierarchy of pitches in major and minor keys. Correlating each of the
24 major and minor profile produced a matrix of pairwise distances
which was fed to a dimensional scaling algorithm. The resulting
geometrical solution was found to have a double circular property
(circle of fifths and relative-parallel relations) which can be
modelled as the surface a 3D torus. Many spatial models of tonality
have this double circular property whether it is implicit
(\citep{Weber1821}, \citep{Schoenberg1948}) or stated explicitly
(\citep{Lerdahl2001}).
\subsection{Set-Class Spaces}
\label{sec-5-4}

Most of these models are limited to description of music in the
major-minor paradigm and are not capable of generalirzing beyond the
``western common practice''. PC-set theory, once again, provides a
possible means to generalise to any kind of pitch-based music. By
considering a collection of tonal objects described by SCs, a
geometric space can be contructued to model their relations based on
some theoreitcal principle. Some PC-set theorists have proposed
explicit geomtric spaces to model relations between SCs. The distances
in these spaces are expressed by models of similarity based on voice
leading (\citep{Cohn2003,Tymoczko2012}) or ICVs and the Fourier
transform (\citep{Quinn2006, Quinn2007}). However, these models are
only designed to represent SCs of one cardinality class at a time and
cannot model the relation between any collection of pitches.

Alternative spatial models are provided by the implicit geometries of
the values produced by the SC similarity measures discussed in chapter
XXXX. Samplaski applied MDS to the values produced from 6 similarity
measures.
\subsection{Time}
\label{sec-5-5}

\begin{itemize}
\item Tonality and time \citep{Martorell2013}
\end{itemize}
\section{SC Similarity Measures}
\label{sec-6}
\subsection{MORRIS}
\label{sec-6-1}
\subsubsection{K}
\label{sec-6-1-1}
\begin{itemize}

\item Details
\label{sec-6-1-1-1}%
\begin{itemize}
\item ICV based
\item Measures similarity
\item Max sim: 55
\item Min sim: 0
\item Average: 10
\item Distinc values: 35
\item 1 to 1 correspondence
\item Castren criteria met: C1, C2, C3.3 C3.4, C4
\item Problems: scale of values not the same for all value groups.
\end{itemize}

\item Formula\\
\label{sec-6-1-1-2}%
$$
k=\frac{\left(\#ICV\left(R\right)+\#ICV\left(S\right)-SIM\left(R,S\right)\right)}{2}
$$ 
Rahn writes it differently (as a funtion of X and Y) 
$$
k\left(X,Y\right)=\frac{\left(\#ICV\left(X\right)+\#ICV\left(Y\right)-SIM\left(X,Y\right)\right)}{2}
$$
or from castren
$$ k(X,Y)= \sum_{i=1}^{6}{MIN(x_{i},y_{i})} $$
\end{itemize} % ends low level
\subsubsection{SIM}
\label{sec-6-1-2}
\begin{itemize}

\item Description\\
\label{sec-6-1-2-1}%
Presented in Morris(1979-80)

\item Details
\label{sec-6-1-2-2}%
\begin{itemize}
\item ICV based
\item 1 to 1 correspondence
\item Measures dissimilarity
\item Max sim: 0
\item Min sim: 65
\item Average: 13
\item Distinc values: 44
\item Castren criteria met: C1, C2, C3.3, C3.4, C4
\item Problems: scale not the same for all value groups. course resolution
  when cardinalityies differ greatly
\end{itemize}

\item Formula\\
\label{sec-6-1-2-3}%
$$ 
SIM\left(X,Y\right)=\sum_{i=1}^{6}\left|x_{i}-y_{i}\right| 
$$ 
or 
$$
SIM\left(X,Y\right)=\#DV\left(ICV\left(X\right),ICV\left(Y\right)\right)
$$
\end{itemize} % ends low level
\subsubsection{ASIM}
\label{sec-6-1-3}
\begin{itemize}

\item Description\\
\label{sec-6-1-3-1}%
Absolute SIM
Scaled version of SIM

\item Details
\label{sec-6-1-3-2}%
\begin{itemize}
\item ICV based
\item 1 to 1 correspondence
\item Max sim: 0
\item Min sim: 1
\item Average: 0.42
\item Distinct values: 79
\item Castren criteria met: C1, C2, C3.1, C3.2, C3.4, C4
\item Problems: Fixed the scale of values, but still coarse resolution
  when cardinalities differ greatly. Scaling is done as the last step.
\end{itemize}

\item Formula\\
\label{sec-6-1-3-3}%
$$
ASIM\left(X,Y\right)=\frac{SIM\left(X,Y\right)}{\left(\#ICV\left(X\right)+\#ICV\left(Y\right)\right)}
$$
\end{itemize} % ends low level
\subsection{LORD}
\label{sec-6-2}
\subsubsection{sf}
\label{sec-6-2-1}
\begin{itemize}

\item Description
\label{sec-6-2-1-1}%
\begin{itemize}
\item Presented in Lord(1981)
\item similar to sim but developed independently
\item sf is subset of SIM
\end{itemize}

\item Details
\label{sec-6-2-1-2}%
\begin{itemize}
\item ICV based
\item 1 to 1 correspondence
\item Same cardinality only
\item Max sim: 0
\item Min sim: 9
\item Average: 3
\item Distinct values: 10
\item Castren criteria met: C3.3, C3.4, C4
\item Problems: Fixed the scale of values, but still coarse resolution
  when cardinalities differ greatly. Scaling is done as the last step.
\end{itemize}

\item Formula\\
\label{sec-6-2-1-3}%
$$ sf\left(X,Y\right)=\frac{\sum_{i=1}^{6}\left|x_{i}-y_{i}\right|}{2} $$
or 
$$ sf\left(X,Y\right)=\frac{\#DV\left(ICV\left(X\right),ICV\left(Y\right)\right)}{2} $$
or
$$ sf(X,Y)=\frac{SIM(X,Y)}{2} $$
\end{itemize} % ends low level
\subsection{TEITELBAUM}
\label{sec-6-3}
\subsubsection{s.i.}
\label{sec-6-3-1}
\begin{itemize}

\item Details
\label{sec-6-3-1-1}%
\begin{itemize}
\item ICV based
\item Same cardinality only
\item 1 to 1 correspondence
\item Z-related sets not compared
\item Max sim: sqrt 2 = 1.41
\item Min sim: sqrt 72 = 8.49
\item Average: 2.85
\item Distinc values: 31
\item Castren criteria met: C3.3 C3.4
\end{itemize}

\item Formula\\
\label{sec-6-3-1-2}%
$$ 
s.i.(X,Y)=\sqrt{\sum_{i=1}^{6}(x_{i}-y_{i})^{2}} 
$$
or
$$ s.i.(X,Y)=\sqrt{ \#DV( ICV(X), ICV(Y) )^{2}  }  $$
\end{itemize} % ends low level
\subsection{ROGERS}
\label{sec-6-4}
\subsubsection{IcVD1}
\label{sec-6-4-1}
\begin{itemize}

\item Description
\label{sec-6-4-1-1}%
\begin{itemize}
\item from Rogers(1992)
\item modification of SIM
\end{itemize}

\item Details
\label{sec-6-4-1-2}%
\begin{itemize}
\item ICV based
\item 1 to 1 correspondence
\item Max sim: 0
\item Min sim: 2
\item Average: 0.59
\item Distinct values: 140
\item Castren criteria met: C1, C2, C3.1, C3.2, C3.4, C4
\item vector components are scaled before being summed
\item Related to \%REL2
\end{itemize}

\item Formula\\
\label{sec-6-4-1-3}%
$$
IcVD_{1}(X,Y)=\sum_{i=1}^{6}{\left|\frac{x_{i}}{\#ICV(X)}+\frac{y_{i}}{\#ICV(Y)}\right|}
$$
\end{itemize} % ends low level
\subsubsection{IcVD2}
\label{sec-6-4-2}
\begin{itemize}

\item Description
\label{sec-6-4-2-1}%
\begin{itemize}
\item from rogers 1992
\item modification of SIM
\item 
\end{itemize}

\item Details
\label{sec-6-4-2-2}%
\begin{itemize}
\item ICV based
\item 1 to 1 correspondence
\item Max sim: 0
\item Min sim: 1.41
\item Average: 0.54
\item Distinc values: 133
\item Castren criteria met: C1, C2, C3.1, C3.2, C3.4
\item Problems: does not produce uniform values for comparable cases
\end{itemize}

\item Formula\\
\label{sec-6-4-2-3}%
$$
IcVD_{2}(X,Y)=\sqrt{\sum{\left(\frac{x_{i}}{\sqrt{\sum{x_{i}^{2}}}}-\frac{y_{i}}{\sqrt{\sum{y_{i}^{2}}}}\right)}^{2}}
$$
\end{itemize} % ends low level
\subsubsection{COS$\theta$}
\label{sec-6-4-3}
\begin{itemize}

\item Description
\label{sec-6-4-3-1}%
\begin{itemize}
\item from rogers (1992)
\item geometric approach
\end{itemize}

\item Details
\label{sec-6-4-3-2}%
\begin{itemize}
\item ICV based
\item 1 to many correspondence
\item Max sim: 1
\item Min sim: 0
\item Average: 0.81
\item Distinct values: 92
\item Castren criteria met: C1, C2, C3.1, C3.2, C3.4
\item Problems: C4
\end{itemize}

\item Formula\\
\label{sec-6-4-3-3}%
$$
Cos\theta(X,Y)=\frac{\sum{x_i.y_i}}{\sqrt{\sum{(x_i)^2}}.\sqrt{\sum{(y_i)^2}}}
$$
\end{itemize} % ends low level
\subsection{ISAACSON}
\label{sec-6-5}
\subsubsection{AMEMB2}
\label{sec-6-5-1}
\begin{itemize}

\item Description
\label{sec-6-5-1-1}%
\begin{itemize}
\item from Isaacson 1990
\item modification of rahns MEMB2
\item divide MEMB2 values by total number of entries in the ICVs of X and
  Y.
\end{itemize}

\item Details
\label{sec-6-5-1-2}%
\begin{itemize}
\item ICV based
\item 1 to 1 correspondence
\item Max sim: 1
\item Min sim: 0
\item Average:
\item Distinct values:
\end{itemize}

\item Formula\\
\label{sec-6-5-1-3}%
$$ AMEMB_{2}=\frac{\sum \left( x_{i}+y_{i} \right)}{\frac{\left(\#X\left(\#X-1\right)+\#Y\left(\#Y-1\right)\right)}{2}} $$
such that $$ \left(x_{i}>0\right) $$ and $$ \left(y_{i}>0\right) $$
\end{itemize} % ends low level
\subsubsection{IcVSIM}
\label{sec-6-5-2}
\begin{itemize}

\item Description
\label{sec-6-5-2-1}%
\begin{itemize}
\item Interval class vector similarity relation
\item the standard deviation of the entries in the idv of two sets
\item Same degree of distinction as s.i.
\item any cardinality
\item Mathematically IcVSIM is a scalled version of s.i.
\end{itemize}

\item Details
\label{sec-6-5-2-2}%
\begin{itemize}
\item ICV based
\item 1 to 1 correspondence
\item Max sim: 0
\item Min sim: 3.64
\item Average: 1.2
\item Distinct values: 121
\item Castren criteria met: C1, C2, C3.4
\end{itemize}

\item Formula\\
\label{sec-6-5-2-3}%
$$ IcVSIM\left(X,Y\right)=\sigma\left(IdV\right) $$
where 
$$ IdV=[(y_{1}-x_{1})(y_{2}-x_{2})...(y_{6}-x_{6})] $$
and
$$ \sigma=\sqrt{\frac{\sum(IdV_{i}-\overline{IdV})^{2}}{6}} $$
where
$$IdV_{i}$$ is the ith term of the interval-difference vector and
$$\overline {IdV}$$ is the average (mean) of the terms in the IdV
Also
$$ s.i.(X,Y)=\sqrt{\sum_{i=1}^{6}{IdV_{i}}^{2}} $$
\end{itemize} % ends low level
\subsubsection{ISIM2}
\label{sec-6-5-3}
\begin{itemize}

\item Description
\label{sec-6-5-3-1}%
\begin{itemize}
\item Saled version of IcVSIM
\item The square root is taken of each term in the ICV
\end{itemize}

\item Details
\label{sec-6-5-3-2}%
\begin{itemize}
\item ICV based
\item 1 to 1 correspondence
\end{itemize}
\end{itemize} % ends low level
\subsubsection{ANGLE}
\label{sec-6-5-4}
\begin{itemize}

\item Description
\label{sec-6-5-4-1}%
\begin{itemize}
\item Same as Cos$\theta$ from Rogers
\end{itemize}

\item Details
\label{sec-6-5-4-2}%
\begin{itemize}
\item ICV based
\end{itemize}

\item Formula\\
\label{sec-6-5-4-3}%
$$ ANGLE(X,Y)= \arccos{\frac{ICV(X) \cdot
ICV(Y)}{|ICV(X)|\times |ICV(Y)|}} $$ 
where the numberator is the dot product of the ICVs 
and the denominator is the product of the magnitudes
\end{itemize} % ends low level
\subsection{RAHN}
\label{sec-6-6}
\subsubsection{AK}
\label{sec-6-6-1}
\begin{itemize}

\item Description
\label{sec-6-6-1-1}%
\begin{itemize}
\item from rahn (1979-80)
\item Absolute (Adjusted) K
\item took morris' k and used it as similarity measure
\item related to ASIM
\end{itemize}

\item Details
\label{sec-6-6-1-2}%
\begin{itemize}
\item ICV based
\item 1 to 1 correspondence
\item Max sim: 1
\item Min sim: 0
\item Average: 0.58
\item Distinct values: 78
\item Castren criteria met: C1, C2, C3.1, C3.2, C3.4, C4
\item Problems: single scale of values (C4), but poor discrimination for
  some value groups.
\end{itemize}

\item Formula\\
\label{sec-6-6-1-3}%
$$ ak\left(X,Y\right)=\frac{2\times k\left(X,Y\right)}{\#ICV\left(X\right)+\#ICV\left(Y\right)} $$
$$ Ak(X,Y)=1-ASIM(X,Y) $$
\end{itemize} % ends low level
\subsubsection{MEMBn}
\label{sec-6-6-2}
\begin{itemize}

\item Description
\label{sec-6-6-2-1}%
\begin{itemize}
\item from rahn (1979-80)
\item Mutual Embedding
\item preliminary step towards other measures
\end{itemize}

\item Details
\label{sec-6-6-2-2}%
\begin{itemize}
\item nCV based
\item 1 to many correspondence
\item one cardinality at a time
\item Max sim: 121
\item Min sim: 0
\item Average: 30
\item Distinc values: 79
\item Castren criteria met: C1, C2, C3.3, C3.4, C4
\item Problems: does not produce uniform scale of values for all value
  groups.
\item by setting n to 2 it compares ICVs
\end{itemize}

\item Formula\\
\label{sec-6-6-2-3}%
$$ MEMB_{n}\left(J,X,Y\right)=EMB\left(J,X\right)+EMB\left(J,Y\right) $$
for all J such that
$$ \#J=n $$
and
$$ EMB\left(J,X\right)>0 $$ and $$ EMB\left(J,Y\right)>0 $$
so\ldots{}
$$ MEMB_{\#X}\left( X,X,Y \right)=EMB\left( X,Y \right) + 1 $$
\end{itemize} % ends low level
\subsubsection{TMEMB}
\label{sec-6-6-3}
\begin{itemize}

\item Description
\label{sec-6-6-3-1}%
\begin{itemize}
\item total mutual embedding
\item from rahn (1979-80)
\item total measure
\end{itemize}

\item Details
\label{sec-6-6-3-2}%
\begin{itemize}
\item nCV based
\item 1 to many correspondence
\item Max sim: 6118
\item Min sim: 0
\item Average: 131
\item Distinct values: 877
\item Castren Criteria met: C1, C2, C3.3, C4, C5
\item Problems: Different value scales for different value groups
\end{itemize}

\item Formula\\
\label{sec-6-6-3-3}%
$$ TMEMB\left(X,Y\right)=\sum_{n=2}^{12}MEMB_{n}\left(J,X,Y\right) $$
\end{itemize} % ends low level
\subsubsection{ATMEMB}
\label{sec-6-6-4}
\begin{itemize}

\item Description
\label{sec-6-6-4-1}%
\begin{itemize}
\item from rahn (1989-80)
\item total measure
\item Absolute (Adjusted) TMEMB
\item ATMEMB is related to TMEMB the same way ASIM is to SIM and AK to K
\end{itemize}

\item Details
\label{sec-6-6-4-2}%
\begin{itemize}
\item nCV based
\item 1 to many correspondence
\item total measures
\item Differentiates between Z-related pairs and I related
\item Max sim: 1
\item Min sim: 0
\item Average: 0.45
\item Distinct values: 101
\item Castren criteria met: C1, C2, C3.1, C3.2, C3.4, C4, C5
\item Problems: Castren says: ``divisor term is flawed\ldots{} high degrees of
  similarity between SCs of clearly different cardinalities. General
  reliability and usefulness is difficulty to determine''
\end{itemize}

\item Formula\\
\label{sec-6-6-4-3}%
$$ ATMEMB\left(X,Y\right)=\frac{TMEMB\left(X,Y\right)}{2^{\#X}+2^{\#Y}-\left(\#X+\#Y+2\right)} $$
\end{itemize} % ends low level
\subsection{LEWIN}
\label{sec-6-7}
\subsubsection{REL}
\label{sec-6-7-1}
\begin{itemize}

\item Description
\label{sec-6-7-1-1}%
\begin{itemize}
\item Presented in Lewin 1979-80
\item probablistic approach to pc-set theory
\item multiplying components has an effect similar to MEMBn (only
  non-zero)
\item by multiplying and sqrting - geometric mean
\end{itemize}

\item Details
\label{sec-6-7-1-2}%
\begin{itemize}
\item nCV based
\item total measure
\item discriminates Z and I related sets
\item 1 to many correspondence
\item Max sim: 1
\item Min sim: 0
\item Average: 0.57
\item Distinct values: 91
\item Castren criteria met: C1, C2, C3.1, C3.2, C3.4, C4, C5
\item Problems
\end{itemize}

\item Formula\\
\label{sec-6-7-1-3}%
$$ REL(A,B)=\frac{\sum_{i=1}^{357}\sqrt{sub(A,i).sub(B,i)}}{\sqrt{\sum_{i=1}^{357}sub(A,i).\sum_{i=1}^{357}sub(B,i)}} $$
or
$$ REL(X,Y)=\frac{ \sum_{A\in TEST}{\sqrt{EMB(A,X) \times EMB(A,Y)}} }{\sqrt{TOTAL(X)\times TOTAL(Y)}}  $$
Were TEST cotains all SCs of cardinality 2 to MIN(\#X,\#Y)
\end{itemize} % ends low level
\subsubsection{REL2}
\label{sec-6-7-2}
\begin{itemize}

\item Description
\label{sec-6-7-2-1}%
\begin{itemize}
\item A specialised version of \ref{sec-6-7-1} that measures only intervallic
  similarity
\item criticises Rahns (x$_{i}$+y$_{i}$) as ``arithmetic awkwardness''
\item Multiplies corresponding IcV entries
\item REL2 increases as corresponding IcV entries increase
\item As cardinality increases, range of REL decreases
\item high level of distinction
\item produces max similarity only when IcVs are identical
\end{itemize}

\item Formula\\
\label{sec-6-7-2-2}%
$$ REL_{2}(X,Y)\frac{2\times\sum\sqrt{(x_{i}y_{i})}}{\sqrt(\#X(\#X-1)\#Y(\#Y-1))} $$
\end{itemize} % ends low level
\subsection{CASTREN}
\label{sec-6-8}
\subsubsection{\%RELn}
\label{sec-6-8-1}
\begin{itemize}

\item Asymmetric Difference vector
\label{sec-6-8-1-1}%

\item nC\%V
\label{sec-6-8-1-2}%

\item Description
\label{sec-6-8-1-3}%
\begin{itemize}
\item Presented in castren
\item Modification of sf but using nC\%V
\item integral part of recrel
\end{itemize}

\item Details
\label{sec-6-8-1-4}%
\begin{itemize}
\item nC\%V based
\item one cardinality at a time
\item 1 to 1 correspondence
\item sometimes discriminate between I related SCsx
\item Max sim: 0
\item Min sim: 100
\item Average: 30
\item Distinct values: 85
\item Castren criteria met: C1, C2, C3.1, C3.2, C3.3, C3.4, C4
\end{itemize}

\item Formula\\
\label{sec-6-8-1-5}%
$$ \%REL_n(X,Y)=\frac{\sum_{i=1}^{p}|x_i-y_i|}{2} $$
where xi and yi are values in the nC\%V and p is the length of the nC\%V.
or
$$ \%REL_{n}\left(X,Y\right)=\frac{\#DV\left(nC\%V(X),nC\%V(Y)\right)}{2} $$
\end{itemize} % ends low level
\subsubsection{T\%REL}
\label{sec-6-8-2}
\begin{itemize}

\item Description
\label{sec-6-8-2-1}%
\begin{itemize}
\item total percentage measure
\item total meaure
\end{itemize}

\item Details
\label{sec-6-8-2-2}%
\begin{itemize}
\item nC\%V based
\item 1 to 1 correspondence
\item total measure
\item discriminated z and I related SCs
\item Max sim: 0
\item Min sim: 100
\item Average: 63
\item Distinct values: 79
\item Castren criteria met: C1, C2, C3.1, C3.2, C3.3, C3.4, C4, C5
\end{itemize}

\item Formula\\
\label{sec-6-8-2-3}%
$$ T\%REL(X,Y)=\frac{\sum_{n=2}^{m}{\%REL_n\left(X,Y\right)}}{m-1} $$
where
if $$ \#X\neq \#Y, m = MIN(\#X, \#Y)$$ else $$ m = \#X-1  $$
\end{itemize} % ends low level
\subsubsection{RECREL}
\label{sec-6-8-3}
\begin{itemize}

\item Comparison Procedure
\label{sec-6-8-3-1}%
\begin{enumerate}
\item Branches
\item Levels
\item Weighting
\item Final Value
\end{enumerate}
\end{itemize} % ends low level
\subsection{BUCHLER}
\label{sec-6-9}
\subsubsection{SATSIM}
\label{sec-6-9-1}
\begin{itemize}

\item Description
\label{sec-6-9-1-1}%
\begin{itemize}
\item orgininally meant to compare ICVs but can be adapted
\end{itemize}

\item SATV
\label{sec-6-9-1-2}%
\begin{itemize}
\item need to define. Saturation vector
\item extent to which an SC is saturated with a subclass
\end{itemize}

\item Formula\\
\label{sec-6-9-1-3}%
$$ 
SATSIM_{n}(X,Y)= \frac{\sum_{i=1}^{p}{\left|SATV_{A}(X)_{i}-SATV_{row}(Y)_{i}\right|+\left|SATV_{A}(Y)_{i}-SATV_{row}(X)_{i}\right|}}{\sum_{i=1}^{p}{\left|SATV_{A}(X)_{i}-SATV_{B}(X)_{i}\right|+\left|SATV_{A}(Y)_{i}-SATV_{B}(Y)_{i}\right|}}
$$
where p is the length of the SATVs
\end{itemize} % ends low level
\subsubsection{CSATSIM}
\label{sec-6-9-2}
\begin{itemize}

\item Description
\label{sec-6-9-2-1}%
\begin{itemize}
\item uses CSATV
\end{itemize}

\item 
%
\end{itemize} % ends low level
\subsubsection{AvgSATSIM}
\label{sec-6-9-3}
\begin{itemize}

\item Formula\\
\label{sec-6-9-3-1}%
$$ AvgSATSIM(X,Y)= \sum_{n=2}^{m-1}{SATSIM_{n}(X,Y)} $$
where $$ m = MIN(\#X,\#Y)  $$
\end{itemize} % ends low level
\subsubsection{TSATSIM}
\label{sec-6-9-4}
\begin{itemize}

\item Description
\label{sec-6-9-4-1}%
\begin{itemize}
\item Extension of SATSIMn
\item quotient of sum of SATSIMn numerators by denominators
\item very close to AvgSATSIM
\item Total measure
\end{itemize}
\end{itemize} % ends low level
\subsection{Comparison Table}
\label{sec-6-10}

\begin{table}[htb]
\caption{Comparison table of similarity measures} 
\begin{center}
\begin{tabular}{llll}
\hline
             &  SIMILARITY  &  VECTOR    &               \\
 THEORIST    &  MEASURE     &  TYPE      &  CARDINALITY  \\
\hline
 TEITELBAUM  &  s.i.        &  INTERVAL  &  SAME         \\
\hline
 LORD        &  sf          &  INTERVAL  &  SAME         \\
\hline
             &  SIM         &  INTERVAL  &  SAME         \\
             &  K           &  INTERVAL  &  SAME         \\
 MORRIS      &  ASIM        &  INTERVAL  &  ANY          \\
\hline
             &  IcVD1       &  INTERVAL  &  ANY          \\
             &  IcVD2       &  INTERVAL  &  ANY          \\
 ROGERS      &  COS         &  INTERVAL  &  ANY          \\
\hline
             &  AMEMB2      &  INTERVAL  &  ANY          \\
             &  IcVSIM      &  INTERVAL  &  ANY          \\
             &  ISIM2       &  INTERVAL  &  ANY          \\
 ISAACSON    &  ANGLE       &  INTERVAL  &  ANY          \\
\hline
             &  AK          &  INTERVAL  &  ANY          \\
             &  MEMBn       &  SUBSET    &  ANY          \\
             &  TMEMB       &  SUBSET    &  ANY          \\
 RAHN        &  ATMEMB      &  SUBSET    &  ANY          \\
\hline
             &  REL2        &  INTERVAL  &  ANY          \\
 LEWIN       &  REL         &  SUBSET    &  ANY          \\
\hline
             &  \%RELn      &  SUBSET    &  ANY          \\
             &  T\%REL      &  SUBSET    &  ANY          \\
 CASTREN     &  RECREL      &  SUBSET    &  ANY          \\
\hline
             &  SATSIM      &  INTERVAL  &  ANY          \\
             &  CSATSIM     &  INTERVAL  &  ANY          \\
             &  TSATSIM     &  SUBSET    &  ANY          \\
 BUCHLER     &  AvgSATSIM   &  SUBSET    &  ANY          \\
\hline
\end{tabular}
\end{center}
\end{table}
\subsection{Comparison Table 2}
\label{sec-6-11}

\begin{table}[htb]
\caption{Comparison table 2 of similarity measures} 
\begin{center}
\begin{tabular}{llllrr}
\hline
             &  SIMILARITY  &  VECTOR  &               &  MEASURE  &        Min  \\
 THEORIST    &  MEASURE     &  TYPE    &  CARDINALITY  &     TYPE  &        Min  \\
\hline
             &  K           &  ICV     &  SAME         &        1  &       0-55  \\
             &  SIM         &  ICV     &  SAME         &        1  &       65-0  \\
 MORRIS      &  ASIM        &  ICV     &  ANY          &        1  &        1-0  \\
\hline
 LORD        &  sf          &  ICV     &  SAME         &        1  &        9-0  \\
\hline
 TEITELBAUM  &  s.i.        &  ICV     &  SAME         &        1  &  8.49-1.41  \\
\hline
             &  IcVD1       &  ICV     &  ANY          &        1  &        2-0  \\
             &  IcVD2       &  ICV     &  ANY          &        1  &             \\
 ROGERS      &  COS         &  ICV     &  ANY          &        1  &             \\
\hline
             &  AMEMB2      &  ICV     &  ANY          &        1  &             \\
             &  IcVSIM      &  ICV     &  ANY          &        1  &             \\
             &  ISIM2       &  ICV     &  ANY          &        1  &             \\
 ISAACSON    &  ANGLE       &  ICV     &  ANY          &        1  &             \\
\hline
             &  AK          &  ICV     &  ANY          &        1  &             \\
             &  MEMBn       &  nCV     &  ANY          &        1  &             \\
             &  TMEMB       &  nCV     &  ANY          &        2  &             \\
 RAHN        &  ATMEMB      &  nCV     &  ANY          &        2  &             \\
\hline
             &  REL2        &  ICV     &  ANY          &        1  &             \\
 LEWIN       &  REL         &  nCV     &  ANY          &        2  &             \\
\hline
             &  \%RELn      &  nCV     &  ANY          &        1  &             \\
             &  T\%REL      &  nCV     &  ANY          &        2  &             \\
 CASTREN     &  RECREL      &  nCV     &  ANY          &        2  &             \\
\hline
             &  SATSIM      &  nCV     &  ANY          &        1  &             \\
             &  CSATSIM     &  ICV     &  ANY          &        1  &             \\
             &  TSATSIM     &  nCV     &  ANY          &        2  &             \\
 BUCHLER     &  AvgSATSIM   &  nCV     &  ANY          &        2  &             \\
\hline
\end{tabular}
\end{center}
\end{table}
\section{Glossary}
\label{sec-7}

\begin{itemize}
\item PC: pitch class
\item set: unordered PC set
\item SC: set class
\item nC: Cardinality class, nC
\item IC: interval class, 2C
\item Tn: transposition
\item I: Inversion
\item TnI: Transposition/Inversion
\item Prime Form: PC set represetning all members of an SC
\item ICV: Interval class vector
\item nCV: n-Subset class vector
\item nC\%V: n-subset class percetage vector
\item SATV: saturation vector
\item CSATV: cyclic saturation vectors
\item DV: difference vector
\item \#X: cardinality, set or vector
\end{itemize}
\section{Test}
\label{sec-8}

\citep{Martorell2013}

\bibliographystyle{plainnat}
\bibliography{/Users/nick/Documents/MendeleyDesktop/library.bib}

\end{document}
