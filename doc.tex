% Created 2014-02-18 Tue 16:24
\documentclass[11pt]{article}
\usepackage[utf8]{inputenc}
\usepackage[T1]{fontenc}
\usepackage{fixltx2e}
\usepackage{graphicx}
\usepackage{longtable}
\usepackage{float}
\usepackage{wrapfig}
\usepackage{rotating}
\usepackage[normalem]{ulem}
\usepackage{amsmath}
\usepackage{textcomp}
\usepackage{marvosym}
\usepackage{wasysym}
\usepackage{amssymb}
\usepackage{hyperref}
\tolerance=1000
\author{NICHOLAS HARLEY}
\date{\today}
\title{SMC RESEARCH PROJECT 2013-2014: Evaluation of Pitch-Class Set Similarity Measures for Tonal Analysis}
\hypersetup{
  pdfkeywords={},
  pdfsubject={},
  pdfcreator={Emacs 24.3.1 (Org mode 8.2.5g)}}
\begin{document}

\maketitle
\tableofcontents


\section{Distance Measures}
\label{sec-1}
\subsection{Euclidean}
\label{sec-1-1}
\subsection{Mahalanobis}
\label{sec-1-2}
\subsection{Correlation}
\label{sec-1-3}
\subsection{Cosine Distance}
\label{sec-1-4}

\section{Comparison Table\hfill{}\textsc{Table}}
\label{sec-2}

\begin{center}
\begin{tabular}{llrll}
Measure & Theorist & Date & Cardinality & Type\\
\hline
\ref{sec-3-7} & Forte & 1973 & SAME & IcV\\
\ref{sec-3-6} & Teitelbaum & 1965 & SAME & IcV\\
\ref{sec-3-16} & Lord & 1981 & SAME & IcV\\
\ref{sec-3-1} & Lewin & 1979 & ANY & SUBSET (Total)\\
\ref{sec-3-2} & Lewin & 1979 & ANY & IcV\\
\ref{sec-3-8} & Morris & 1979 & ANY* & IcV\\
\ref{sec-3-9} & Morris & 1979 & ANY & IcV\\
\ref{sec-3-10} & Morris & 1979 & SAME & IcV\\
\ref{sec-3-11} & Morris & 1979 & ANY & IcV\\
\ref{sec-3-12-2-2} & Rahn & 1979 & SAME & SUBSET\\
\ref{sec-3-13} & Isaacson* & 1990 & ANY & SUBSET*\\
\ref{sec-3-15} & Rahn & 1979 & ANY & SUBSET (Total)\\
\ref{sec-3-4} & Isaacson & 1990 & ANY & IcV\\
\ref{sec-3-5} & Isaacson & 1996 & ANY & IcV\\
\ref{sec-3-26} & Rogers & 1992 & ANY & IcV\\
\ref{sec-3-27} & Rogers & 1992 & ANY & IcV\\
\ref{sec-3-28} & Rogers & 1992 & ANY & IcV\\
\ref{sec-3-18} & Castren & 1994 & ANY & SUBSET\\
\texttt{\%REL2} & Castren & 1994 & ANY & IcV\\
\texttt{T\%REL} & Castren & 1994 & ANY & SUBSET (Total)\\
\ref{sec-3-21} & Castren & 1994 & ANY & SUBSET (Total)\\
\ref{sec-3-17} & Scott \& Isaacson & 1999 & ANY & IcV\\
\ref{sec-3-22} & Buchler & 1998 & ANY & IcV-SATV\\
\ref{sec-3-23} & Buchler & 1998 & ANY & IcV-SATV\\
\ref{sec-3-25} & Buchler & 1998 & ANY & SUBSET-SATVn (Total)\\
\ref{sec-3-24} & Buchler & 1998 & ANY & SUBSET-SATVn (Total)\\
\ref{sec-3-3} & Lewin & 1977 & - & SUBSET\\
\end{tabular}
\end{center}
\section{Theoretical Resemblace Models\hfill{}\textsc{SimilarityMeasures}}
\label{sec-3}
\subsection{REL\hfill{}\textsc{Lewin}}
\label{sec-3-1}
FROM: \texttt{Lewin}
\subsubsection{Description}
\label{sec-3-1-1}
\begin{itemize}
\item For each class being compare the subset vector is computed
\item the subset vector is 357 dimensions
\begin{itemize}
\item 6 for IcV then 351 for each correspondin to the embedding all
Tn related sets
\end{itemize}
\item sub(X,i) is the ith term in the subset vector
\end{itemize}

\subsubsection{Formula}
\label{sec-3-1-2}
$$ REL ( A,B ) = \frac{\sum_{i=1}^{357}\sqrt{sub(A,i).sub(B,i)} }{\sqrt{\sum_{i=1}^{357}sub(A,i).\sum_{i=1}^{357}sub(B,i)}} $$

\subsection{REL2\hfill{}\textsc{Lewin}}
\label{sec-3-2}
FROM: \texttt{Lewin}, 1979-80
\subsubsection{Description}
\label{sec-3-2-1}
\begin{itemize}
\item A specialised version of \ref{sec-3-1} that measures only intervallic
similarity
\item criticises Rahns (x$_{\text{i}}$+y$_{\text{i}}$) as "arithmetic awkwardness"
\item Multiplies corresponding IcV entries
\item REL2 increases as corresponding IcV entries increase
\item As cardinality increases, range of REL decreases
\item high level of distinction
\item produces max similarity only when IcVs are identical
\end{itemize}

\subsubsection{Formula}
\label{sec-3-2-2}
$$ REL_{2}(X,Y) \frac{2\times \sum \sqrt{(x_{i}y_{i})}}{\sqrt(\#X(\#X-1)\#Y(\#Y-1))} $$

\subsection{EMB\hfill{}\textsc{Lewin}}
\label{sec-3-3}
\subsubsection{Description}
\label{sec-3-3-1}
\begin{itemize}
\item EMB(A,X) is the number of pitch-class sets in set-class A which
are subsets of a given pitch-class set in set-class X.
\end{itemize}
\subsection{IcVSIM\hfill{}\textsc{Isaacson}}
\label{sec-3-4}
From: \texttt{Isaacson, 1990}
\subsubsection{Description}
\label{sec-3-4-1}
\begin{itemize}
\item the standard deviation of the entries in the \ref{sec-21-19} of two sets
\item Same degree of distinction as \texttt{s.i.}
\item any cardinality
\item Mathematically IcVSIM is a scalled version of s.i.
\end{itemize}

\subsubsection{Formula}
\label{sec-3-4-2}
$$ IcVSIM\left(X,Y\right)=\sigma\left(IdV\right) $$
where 
$$ IdV=[(y_{1}-x_{1})(y_{2}-x_{2})...(y_{6}-x_{6})] $$
and
$$ \sigma =  \sqrt {\frac {\sum (IdV_{i}- \overline {IdV})^{2}}{6}} $$
where
$$IdV_{i}$$ is the ith term of the interval-difference vector and
$$\overline {IdV}$$ is the average (mean) of the terms in the IdV

\subsection{ISIM2\hfill{}\textsc{Isaacson}}
\label{sec-3-5}
FROM: \texttt{Isaacson}
\subsubsection{Description}
\label{sec-3-5-1}
\begin{itemize}
\item IcV entries are scaled by taking the square root
\item otherwise its the same as \ref{sec-3-4}
\item \texttt{Samplaski} found ISIM2 to be inconsistent with itself with \ref{sec-22}
\end{itemize}

\subsection{s.i.\hfill{}\textsc{Teitelbaum}}
\label{sec-3-6}
\subsubsection{Description}
\label{sec-3-6-1}
\begin{itemize}
\item Greater discrimincation than Forte, Morris, Lord
\item same cardinality
\item vlaues approach 0 when the pcsets being compared are in a close
intervallic relationship
\item high level of distinction (almost as high as \texttt{Lewin}'s \ref{sec-3-2})
\item can compare the intervallic similarity of sets with the same
cardinality to a reference
\item evaluable to some degree by MDS, but not very attractive
\end{itemize}

\subsubsection{Formula}
\label{sec-3-6-2}
$$  s.i.(X,Y) = \sqrt{\sum (x_{i}-y_{i})^{2}} $$

\subsection{R-relations\hfill{}\textsc{Forte}}
\label{sec-3-7}
\begin{itemize}
\item Rn
\item Same cardinality
\item not a complete system (many IcV pairs have no Rn relation
\item See \texttt{Isaacson} for criticism
\end{itemize}

\subsection{SIM\hfill{}\textsc{Morris}}
\label{sec-3-8}
From: \texttt{Morris}
\subsubsection{Description}
\label{sec-3-8-1}

\begin{itemize}
\item sum of the absolute values of the differences between
corresponding IcV entries for the sets
\item Produces 15 unique values:
\begin{itemize}
\item four each for trichords and tetrachords
\item three each for pentachords and hexachords
\end{itemize}
\end{itemize}
\subsubsection{Formula}
\label{sec-3-8-2}
$$ SIM \left ( X, Y \right ) = \sum_{i = 1}^{6} \left | x_{i} - y_{i} \right | $$   

\subsection{ASIM\hfill{}\textsc{Morris}}
\label{sec-3-9}
From: \texttt{Morris}
\subsubsection{Description}
\label{sec-3-9-1}
\begin{itemize}
\item Any size/cardinality:
\begin{itemize}
\item but there is "diminishing return" as the difference between the
cardinalities of the two sets increases
\item values of SIM increase as the difference in cardinality between
sets increases
\item ASIM(X,Y): values are weighted by dividing them by the sum of
the cardinalities of the sets
\end{itemize}
\end{itemize}
\subsubsection{Formula}
\label{sec-3-9-2}

$$ ASIM \left ( X, Y \right ) = \frac{SIM \left ( X, Y \right )}{\left ( \# V \left (X \right ) + \# V \left (Y \right ) \right )} $$

\subsection{K\hfill{}\textsc{Morris}}
\label{sec-3-10}
From: \texttt{Morris}
\subsubsection{Description}
\label{sec-3-10-1}

\begin{itemize}
\item Number of ics in common between the sets X and Y
\item It is a function of SIM(X,Y)
\item Rahn prefers k to SIM
\end{itemize}
\subsubsection{Formula}
\label{sec-3-10-2}

$$ k = \frac{\left( \# V  \left( R \right) + \# V  \left( S \right) - SIM  \left( R, S \right) \right)}{2} $$
Rahn writes it differently (as a funtion of X and Y)
$$ k\left(X, Y \right) = \frac{\left( \# V  \left( R \right) + \# V  \left( S \right) - SIM  \left( R, S \right) \right)}{2} $$

\subsection{AK\hfill{}\textsc{Rahn}}
\label{sec-3-11}
From: \texttt{Rahn}
\subsubsection{Description}
\label{sec-3-11-1}
\begin{itemize}
\item absolute or adjusted \hyperref[sec-3-10]{k} (from Morris, comparable to \ref{sec-3-9})
\item ASIM and Ak are closeley relate
\end{itemize}

\subsubsection{Fromula}
\label{sec-3-11-2}
$$ ak \left( X, Y \right) = \frac{2 \times k \left( X, Y \right)}{\# V \left( X \right) + \# V \left( Y \right) }  $$
$$ Ak(X,Y)=1-ASIM(X,Y) $$

\subsection{MEMB\hfill{}\textsc{Rahn}}
\label{sec-3-12}
FROM: \texttt{Rahn}
\subsubsection{Description}
\label{sec-3-12-1}
\begin{itemize}
\item \ref{sec-3-12-2-1} counts number of subsets X of size n
\item a subset must be present in both X and Y before it is counted
\item by setting n to 2 you get \ref{sec-3-12-2-2} (measure of ic similarity)
\item works badly when cardinalities differ greatly
\end{itemize}

\subsubsection{Formula}
\label{sec-3-12-2}
\begin{enumerate}
\item MEMBn
\label{sec-3-12-2-1}
$$ MEMB_{n} \left( J,X,Y \right) = EMB \left( J,X \right) + EMB \left( J,Y \right)  $$
for all J such that
$$ \# J = n $$
and
$$ EMB \left( J,X \right) > 0 $$
and 
$$ EMB \left( J,Y \right) > 0 $$
so\ldots{}
$$ MEMB_{\#X}\left( X,X,Y \right)=EMB\left( X,Y \right) + 1 $$
\item MEMB2
\label{sec-3-12-2-2}
$$ MEMB_{2}\left(X,Y\right)=\sum_{i=1}^{6}\left(x_{i}+y_{i}\right) $$
such that $$ \left(x_{i}>0\right) $$ and $$ \left(y_{i}>0\right) $$
\end{enumerate}

\subsection{AMEMB2\hfill{}\textsc{Isaacson}}
\label{sec-3-13}
From: \texttt{Isaacson}
\subsubsection{Description}
\label{sec-3-13-1}
\begin{itemize}
\item Isaacson describes a scaled version of \ref{sec-3-12-2-2}
\item Applies a normalisation factor equivelant to that used by \texttt{Rahn}
to derive \ref{sec-3-15} from \ref{sec-3-14}
\item range of values decreases as cardinality increases
\item AMEMB increases as cardinality increases - troubling
\end{itemize}

\subsubsection{Formula}
\label{sec-3-13-2}
$$ AMEMB_{2}=\frac{\sum \left( x_{i}+y_{i} \right)}{\frac{\left(\#X\left(\#X-1\right)+\#Y\left(\#Y-1\right)\right)}{2}} $$
such that $$ \left(x_{i}>0\right) $$ and $$ \left(y_{i}>0\right) $$

\subsection{TMEMB\hfill{}\textsc{Rahn}}
\label{sec-3-14}
From: \texttt{Rahn}
\subsubsection{Description}
\label{sec-3-14-1}
\begin{itemize}
\item the sum of all \ref{sec-3-12-2-1} (n = 2 to 12)
\item distinguishes between Z-related sets
\end{itemize}

\subsubsection{Formula}
\label{sec-3-14-2}
$$ TMEMB \left( A,B \right) = \sum_{n=2}^{12}MEMB_{n}\left( X,A,B \right) $$

\subsection{ATMEMB\hfill{}\textsc{Rahn}}
\label{sec-3-15}
FROM: \texttt{Rahn}
\subsubsection{Description}
\label{sec-3-15-1}
\begin{itemize}
\item Absolute/adjjusted version of \ref{sec-3-14}
\item distinguishes between Z-related sets
\end{itemize}

\subsubsection{Formula}
\label{sec-3-15-2}

$$ ATMEMB\left(A,B\right)=\frac{TMEMB\left(A,B\right)}{2^{\#A}+2^{\#B}-\left(\#A+\#B+2\right)} $$

\subsection{sf\hfill{}\textsc{Lord}}
\label{sec-3-16}
From: \texttt{Lord}
\subsubsection{Description}
\label{sec-3-16-1}
\begin{itemize}
\item half sum of absolute values of the differences between
corresponding IcV entries of the sets
\item sf is a subset of \ref{sec-3-8}
\item Same cardinality
\item Lords values can be inferred from \texttt{Morris}'
\end{itemize}

\subsubsection{Formula}
\label{sec-3-16-2}
$$  sf \left ( X,Y \right ) = \frac{ \sum_{i = 1}^{6} \left | x_{i} - y_{i} \right |}{2}  $$
where X and Y are any pcset from 3 to 9 notes and
$$ x_{i} = IcV(X)_{i} $$ and $$ y_{i} = IcV(Y)_{i} $$

\subsection{ANGLE\hfill{}\textsc{Isaacson}}
\label{sec-3-17}
\subsubsection{Description}
\label{sec-3-17-1}

\subsection{\%RELn\hfill{}\textsc{Castren}}
\label{sec-3-18}
FROM: \texttt{Castren}
\subsubsection{Description}
\label{sec-3-18-1}
\begin{itemize}
\item compare proportionate subset-class contents of two set-classes
\item uses \ref{sec-21-21}
\end{itemize}

\subsubsection{Formula}
\label{sec-3-18-2}
$$ \%REL_n(X,Y)=\frac{\sum_{i=1}^{p}|x_i-y_i|}{2} $$
where xi and yi are values in the \ref{sec-21-21}

\subsection{\%REL2\hfill{}\textsc{Castren}}
\label{sec-3-19}
FROM: \texttt{Castren}
\subsubsection{Description}
\label{sec-3-19-1}
\begin{itemize}
\item is castrens modification of \texttt{Lord}'s \ref{sec-3-16}
\item \ref{sec-3-18} with n = 2
\item thus measures intervallic similarity
\end{itemize}

\subsection{T\%REL\hfill{}\textsc{Castren}}
\label{sec-3-20}
FROM: \texttt{Castren}
\subsubsection{Description}
\label{sec-3-20-1}
\begin{itemize}
\item Total percentage RELation
\item is the arithmetic mean of all \%RELn values for n=2 to min(\#X,\#Y)
\item Considered to be a preliminary version of \ref{sec-3-21}
\item Total measure
\end{itemize}

\subsection{RECREL\hfill{}\textsc{Castren}}
\label{sec-3-21}
FROM: \texttt{Castren}
\subsubsection{Description}
\label{sec-3-21-1}
\begin{itemize}
\item examines the similarity between two set-classes by composing a
net of pairings of all embeddable subset-classes, both shared
and non- shared.
\item RECREL evaluates function \texttt{\%RELn} many times during the process.
\item The final RECREL value is the arithmetic mean of the individual
\%RELn values.
\end{itemize}

\subsection{SATSIM\hfill{}\textsc{Buchler}}
\label{sec-3-22}
\subsubsection{Description}
\label{sec-3-22-1}
\begin{itemize}
\item SATuration SIMilarity index
\item based on interval-class saturation vectors (\ref{sec-21-20})
\end{itemize}

\subsection{CSATSIM\hfill{}\textsc{Buchler}}
\label{sec-3-23}
\subsubsection{Description}
\label{sec-3-23-1}
\begin{itemize}
\item extension of \ref{sec-3-22}
\end{itemize}

\subsection{TSATSIM\hfill{}\textsc{Buchler}}
\label{sec-3-24}
\subsubsection{Description}
\label{sec-3-24-1}
\begin{itemize}
\item Total subset SATuration SIMilarity index
\item calculated by dividing the sum of the numerators of all SATSIMn
comparisons by the sum of the denominators
\item very similar to \ref{sec-3-25}
\end{itemize}

\subsection{AvgSATSIM\hfill{}\textsc{Buchler}}
\label{sec-3-25}
\subsubsection{Description}
\label{sec-3-25-1}
\begin{itemize}
\item based on subset-class saturation vectors
\item first calculate SATSIMn values (cardinality class n SATuration
SIMilarity), n reaching
\end{itemize}
from 2 to m-1 (m = min[\#X,\#Y]).
\begin{itemize}
\item The SATSIMn comparisons are made similarly to the comparisons in
SATSIM. The final AvgSATSIM value is the arithmetic mean of the
individual SATSIMn values
\end{itemize}

\subsection{IcVD1\hfill{}\textsc{Rogers}}
\label{sec-3-26}
\subsubsection{Description}
\label{sec-3-26-1}
\begin{itemize}
\item Modification of \texttt{Morris}'s \ref{sec-3-8}
\end{itemize}

\subsubsection{Formula}
\label{sec-3-26-2}
$$\%REL_2(X,Y)=IcVd_1(X,Y)\times50 $$

\subsection{IcVD2\hfill{}\textsc{Rogers}}
\label{sec-3-27}
\subsubsection{Description}
\label{sec-3-27-1}
\begin{itemize}
\item use IcV like geometric vectors is 6D space
\item IcVD2 is the distance between the ends of the two vectors
\item the IcVs are normalised
\end{itemize}

\subsubsection{Formula}
\label{sec-3-27-2}
$$ IcVD_2(X,Y)=\sqrt{\sum{( \frac{x_i}{\sqrt{\sum(x_i)^2}}}-\frac{y_i}{\sqrt{\sum(y_i)^2}})^2} $$

\subsection{cos(-)\hfill{}\textsc{Rogers}}
\label{sec-3-28}
\subsubsection{Description}
\label{sec-3-28-1}
\begin{itemize}
\item is cos of the angle between the normalised IcVs
\end{itemize}

\subsubsection{Formula}
\label{sec-3-28-2}
$$ Cos\theta(X,Y)=\frac{\sum{x_i.y_i}}{\sqrt{\sum{(x_i)^2}}.\sqrt{\sum{(y_i)^2}}} $$

\section{Forte, 1973\hfill{}\textsc{Rn}}
\label{sec-4}
\begin{itemize}
\item \texttt{The Structure of Atonal Music}
\item \ref{sec-3-7}
\end{itemize}

\section{Lord, 1981\hfill{}\textsc{sf}}
\label{sec-5}
\begin{itemize}
\item \texttt{Intervallic Similarity Relations in Atonal Set Analysis}
\item \ref{sec-3-16}
\end{itemize}

\section{Morris, 1979\hfill{}\textsc{SIM:ASIM:K}}
\label{sec-6}
\begin{itemize}
\item \texttt{A similarity index for pitch-class sets}
\item \ref{sec-3-8}, \ref{sec-3-9}, \ref{sec-3-10}
\end{itemize}

\section{Isaacson, 1990\hfill{}\textsc{IcVSIM:AMEMB2}}
\label{sec-7}
\begin{itemize}
\item \texttt{Similarity of Interval-Class Content Between Pitch-Class Sets:
    The IcVSIM Relation}
\item Isaacson Suggests 3 criteria for a similarity measure
\begin{enumerate}
\item provide a distinct value for every pair of sets
\item be useful (not just usable) for sets of any size
\item provide a wide range of discrete values
\end{enumerate}
\item on the basis of these criteria he finds the measures of
Teitelbaum, Forte, Morris, Rahn, Lewin, Lord to be inadequate
\item Proposes \ref{sec-3-4}
\end{itemize}

\section{Castren, 1994\hfill{}\textsc{RECREL:%REL}}
\label{sec-8}
\section{Buchler, 1997}
\label{sec-9}
\begin{itemize}
\item \texttt{relative saturation of subsets and interval cycles as a means for
    determining set-class similarity}
\item Contains \ref{sec-3-25}
\item Describes \ref{sec-3-21}
\end{itemize}

\section{Isaacson, 1996\hfill{}\textsc{ISIM2}}
\label{sec-10}
\begin{itemize}
\item \texttt{Issues in the study of similarity in atonal music}
\item Good discussion of similarity over all
\item \ref{sec-3-5}
\item weighted version of IcVSIM
\end{itemize}

\section{Lewin, 1979-80\hfill{}\textsc{REL:REL2}}
\label{sec-11}
\begin{itemize}
\item \texttt{A Response to a Response: On PCSet Relatedness}
\item \ref{sec-3-1}, \ref{sec-3-2}
\end{itemize}

\section{Rahn, 1979\hfill{}\textsc{MEMB:TMEMB:ATMEMB}}
\label{sec-12}
\begin{itemize}
\item \texttt{Relating Sets}
\item \ref{sec-3-12}, \ref{sec-3-14}, \ref{sec-3-15}
\end{itemize}

\section{Scott \& Isaacson, 1998\hfill{}\textsc{ANGLE}}
\label{sec-13}
\begin{itemize}
\item \textbf{The Interval Angle: A Similarity Measure for Pitch-Class Sets}
\item ANGLE
\item STATEMENT 11
\begin{itemize}
\item can be generalized for figured-bass
\item The new construction can distinguish between
\end{itemize}
\end{itemize}
major and minor chords and between different doublings and different
inversions of the chords
\begin{itemize}
\item STATEMENT 12
\begin{itemize}
\item ANGLE M
\item further extension to ANGLE
\end{itemize}
\end{itemize}

\section{Teitelbaum, 1965\hfill{}\textsc{si}}
\label{sec-14}
\begin{itemize}
\item \texttt{Intervallic Relations in Atonal Music}
\item \ref{sec-3-6}
\end{itemize}

\section{Samplaski, 2005\hfill{}\textsc{MDS}}
\label{sec-15}
\texttt{Mapping the Geometries of Pitch-Class Set Similarity Measures via
Multidimensional Scaling}
\subsection{methodology}
\label{sec-15-1}
\begin{itemize}
\item 6 pcset similarity measures investigated
\begin{itemize}
\item \textbf{3 interval based}
\begin{enumerate}
\item \texttt{ANGLE}
\item \texttt{IcVSIM}
\item \texttt{ISIM2}
\end{enumerate}
\item \textbf{3 subset based} use subset embedding
\begin{enumerate}
\item \texttt{RECREL}
\item \ref{sec-3-15}
\item \texttt{AMEMB}
\end{enumerate}
\end{itemize}
\item \textbf{3 cardinalities} under \hyperref[sec-21-7]{Tn/I}-equivelance
\begin{itemize}
\item trichords: 3
\item tetrachords: 4
\item pentachords: 5
\item ratings for each cardinality of set-class separately as well as
contiguously grouped together (3+4, 4+5, 3+4+5) were studied
\end{itemize}
\item \ref{sec-22} applied to matrices of (dis)similarities
\item based on goodness-of-fit analysis\ldots{}
\begin{itemize}
\item four-dimensional geometric solutions were found for the
icv-based measure
\item five-dimensional solutions were found for the subset based
measures
\end{itemize}
\end{itemize}

\subsection{motivation}
\label{sec-15-2}
\begin{enumerate}
\item Visualisation
\begin{itemize}
\item similarity measures yield alot of data
\end{itemize}
\item not satisfied with \texttt{scott and isaacoson} conclusions about
correlation
\begin{itemize}
\item correlations (as single, all-subsuming numbers for pairs of
measures) do nothing to show what constructs might underlie
the ratings being produced.
\end{itemize}
\item geometric visualisation allows a "reality check"
\begin{itemize}
\item there might be problems with a measure's numerical ratings
that are not evident from inspection of them
\end{itemize}
\item \texttt{see here}
\end{enumerate}

\subsection{Conclusions}
\label{sec-15-3}
The overall results are generally consistent with the idea that
these functions all measure constructs relating to familiar scales
(diatonic, hexatonic, octatonic, etc.). The results are also
compared with several systems of pcset genera. ISIM2 was found to
be inconsistent with itself in terms of the geometries it
produced. Several set-classes had coordinates near zero along
various dimensions in the derived configurations, indicating that
in a formal quantitative sense they do not possess the
corresponding musical properties being measured; this may raise
questions concerning the relative aesthetic worth of some such
set-classes.

\subsection{more}
\label{sec-15-4}
\begin{itemize}
\item \ref{sec-22-11} paragraph 18-19
\item \ref{sec-21-17} paragraph 20, paragraphs 52-53
\item \ref{sec-22-12} note
\end{itemize}

\section{Forte, 1988\hfill{}\textsc{genera}}
\label{sec-16}
\begin{itemize}
\item \texttt{Pitch-Class Set Genera and the Origin of Modern Harmonic Species}
\item \texttt{Samplaski} paragraph 58
\end{itemize}

\section{Parks, 1989\hfill{}\textsc{genera}}
\label{sec-17}
\begin{itemize}
\item \texttt{The Music of Claude Debussy}
\item \texttt{Samplaski} paragraph 59-60
\end{itemize}

\section{Quinn, 1997\hfill{}\textsc{genera:CA}}
\label{sec-18}
\begin{itemize}
\item \texttt{On Similarity, Relations, and Similarity Relations}
\item \texttt{Samplaski} paragraph 61
\item used \ref{sec-22-11} on ratings from similarity measures
\item found correspondence among measures
\item defined 8 genera
\begin{itemize}
\item there were some "fence sitters"
\item argued strongly for a fuzzy set theory of pcset similarity
\end{itemize}
\end{itemize}

\section{Quinn, 2001\hfill{}\textsc{CA}}
\label{sec-19}
\begin{itemize}
\item \texttt{Listening to similarity relations}
\item What constitutes a good similarity measure?
\begin{enumerate}
\item The ways in which we are accustomed to talking about similarity
relations are not as productive as they seem to be, and there
are better ways to do it.
\item Comparison of various similarity relations from such a
different point of view shows that they are more related to
each other, and to a lot of other theory, than they appear to
be in traditional modes of discourse.
\end{enumerate}
\item \texttt{Samplaski} paragraphs 62-63
\begin{itemize}
\item found clusters using monte carlo analysis
\end{itemize}
\end{itemize}

\section{Regener, 1974}
\label{sec-20}
\begin{itemize}
\item \texttt{On Allen Forte's Theory of Chords}
\item 
\end{itemize}

\section{GLOSSARY}
\label{sec-21}
\subsection{pc}
\label{sec-21-1}
PITCH-CLASS. A set of all pitches that are enharmonically identical
and/or related by any number of octaves. There are twelve pcs,
numbered from 0 to 11. pc 0 contains all C naturals, all B sharps, all
D double-flats; pc 1 contains all Dbs, all C\#s; pc 2 contains all Ds,
all C\#\#s, all Ebbs, and so forth-pc 11 which contains all Bs, A\#\#s,
and Cbs.  From \texttt{Morris}

\subsection{ic}
\label{sec-21-2}
INTERVAL CLASS. A set of all interval that differ by multiples of 12
semi-tones and/or are complementary respect with to the octave. There
are six ics, numbered from 1 to 6. ic 1 contains all minor 2nds, all
major 7ths, all diminished octaves, all augmented 8ves, all min 9ths,
etc.; ic 2 contains all major 2nds, all dim 3rds, all aug 6ths, all
min 7ths, all maj 9ths,etc. -and so forth-ic 6 contains all tritones,
and intervals of a tritone plus any amount of octaves. The ic may also
be defined as the set of intervals between any of the members of one
pc and any of the members of another. The intervals between any D \#
and any F are all members of ic 2.  From \texttt{Morris}

\subsection{set}
\label{sec-21-3}
An unordered collection of pcs without replication. The set (0,3,4) is
the same as (0,4,3) and (4,3,0), etc. A particular set may be denoted
by a capital letter. For instance, T = (0,3,4). There are 4,096
distinct sets.  From \texttt{Morris}

\subsection{SC}
\label{sec-21-4}
SET-CLASS. A collection of sets related to one another by Tn and/or I.
From \texttt{Morris}

\subsection{Tn}
\label{sec-21-5}
Tn TRANSPOSITION by 'n' semitones 'higher'. To transpose a set we add
n to each pc in the set; if the sum exceeds 11, we reduce it by 12. If
the set (8,5,7) is subjected to T5, the result is 8+5, 5+5, and 7+5 or
(1,10,0). If W=(8,5,7), then (1,10,0) may be written T5W.  From \texttt{Morris}

\subsection{I}
\label{sec-21-6}
INVERSION. An operation on pcs which sends, 1 to 11 and vice- versa, 2
to 10 and the reverse, 3 to 9, 4 to 8, 5 to 7, while 0 and 6 remain
the same. The inversion of the set (6,8,4,3) is (6,4,8,9) (and
vice-versa). If (6,8,4,3) = D, then ID = (6,4,8,9).  From \texttt{Morris}

\subsection{TnI}
\label{sec-21-7}
Inversion followed by transposition. T5I of (7,9.4) is produced by
taking the inversion which is (5,3,8) and transposing it by T5 which
results in (10,8,1). If our original set is L, then T5IL= (10,8,1).
From \texttt{Morris}

\subsection{Invariance}
\label{sec-21-8}
A set is invariant if it remains unchanged after transformation under
Tn or I or both. T4I of the set H which is (3,1,8) results in
invariances since 3 becomes 9+4 or 1, 1 becomes 11+4 or 3, and 8
becomes 4+4 or 8. We say that our set is invariant under T4I. T4H= H
From \texttt{Morris}

\subsection{V}
\label{sec-21-9}
INTERVAL-CLASS-VECTOR. A listing of the amount of ics of each type in
a particular set. V(Y) is the interval-class-vector associated with
the set Y. A V is an array of six numbers square brackets. The first
number is called V1 and gives the amount of ic Is in the set; the
second number or argument, V2, is the number of ic 2s in the set;
etc., to the sixth argument, V6, which gives the number of ic 6s. In
the expression, V(B) = [1,1,1,0,0,0], we are asserting the set B has
one ic 1, one ic 2, one ic 3, and no ic 4, 5, or 6. The set (5,7,8)
could be B. We may determine the V of any set by examining all pairs
of pcs in the set, finding the ic for each pair, and registering it in
the appropriate argument in the array. If the set R is (8,4,2,0) the
ic for 8 and 4 is 4; the ic for 8 and 2 is 6; the ic for 8 and 0 is 4;
the ic for 4 and 2 is 2; for 4 and 0 we have 4 and for 2 and 0 we
have 2. Thus, V(R) = [0,2,0,3,0,1].  From \texttt{Morris}

\subsection{\#R}
\label{sec-21-10}
Where R is a set, \#R denotes its cardinality, is, the number that of
pcs in R.  From \texttt{Morris}

\subsection{\#(V)R}
\label{sec-21-11}
The number of ics in V(R). Where R contains n pcs (n = \#R), 4V(R) is
equal to the sum of whole numbers starting with 1 and ending with
(n-1). A set of seven pcs has 1+2+3+4+5+6 or 21 ics.  From \texttt{Morris}

\subsection{Membership $\in$}
\label{sec-21-12}
5 $\in$ (8,5,7) d $\in$ R (a pc named d is a member of the set R).  From
\texttt{Morris}

\subsection{Inclusion $\subset$}
\label{sec-21-13}
R $\subset$ T if every pc in set R is also in set T (R is a sub-set of
T).  (7,5,8) $\subset$ (8,5,3,7,0) From \texttt{Morris}

\subsection{| x | (Absolute Value)}
\label{sec-21-14}
Take the positive sign of the expression enclosed in Is. | 5 | = 5; |
-7 | = (+)7; | 7-2 | = | 2-7 |.  From \texttt{Morris}

\subsection{Genera}
\label{sec-21-15}
\begin{itemize}
\item classical
\begin{itemize}
\item an object either belongs or does not
\end{itemize}
\item fuzzy
\begin{itemize}
\item an object has a probability of belonging
\end{itemize}
\item many genera systems have been proposed
\begin{itemize}
\item Ericksson, 1986
\item \texttt{Forte, 1988}
\item \texttt{Parks, 1989}
\item \texttt{Quinn, 1997}
\item \texttt{Quinn, 2001}
\end{itemize}
\item Some dont relate directly to pcsets: e.g., Hanson, 1960; Harris,
1989; Hindemith, 1937/42; Wolpert, 1951, 1972
\end{itemize}

\subsection{Z-Relation}
\label{sec-21-16}
\begin{itemize}
\item Same icv but not related by TnI
\end{itemize}

\subsection{robustness}
\label{sec-21-17}
\begin{itemize}
\item whether changing the set of objects being compared alters the
perceived/computed similarity between the original set of objects.
\item "yields consistent relative MDS geometries, save for scaling,
regardless of surrounding context" \texttt{samp} paragraph 21
\item \texttt{Samplaski} tested robustness of similarity measures by analysing
cardinalities in isolation as well as combinations of contiguous
cardinalities
\end{itemize}

\subsection{Significance}
\label{sec-21-18}
\begin{itemize}
\item the probability that the observed result might have occurred by
chance
\item \texttt{Samplaski} paragraph 27
\end{itemize}

\subsection{IdV}
\label{sec-21-19}
\begin{itemize}
\item Interval-difference Vector
\item the difference between the terms of 2 ic-vectors
\end{itemize}

\subsection{SATV}
\label{sec-21-20}
\begin{itemize}
\item Bruchler uses them in \ref{sec-3-22}
\item derived by comparing the number of instances of each
interval-class in a set-class with both the minimum and the
maximum number of the corresponding interval-class instances that
can be found in any set-class of the same cardinality
\item From a saturation vector one can thus see the degree of saturation
of each interval-class vector component.
\item Kussi Appendix 2
\end{itemize}

\subsection{nC\%V}
\label{sec-21-21}
\begin{itemize}
\item n-class percentage vector
\item modification of a \ref{sec-21-22} from castren
\item used in \ref{sec-3-18}
\end{itemize}

\subsection{nCV}
\label{sec-21-22}
\begin{itemize}
\item array of numbers corresponding to EMB(A,X)
\item with A running through all set-classes in cardinality class n
\end{itemize}

\section{MDS}
\label{sec-22}
\subsection{non-metric MDS}
\label{sec-22-1}
\begin{itemize}
\item Shepard (1962), Kruskal (1964a, 1964b)
\item assumes that the distance or proximity values of the matrix are
directly related by some unknown function to distances between
the objects in some underlying abstract N-dimensional Euclidean
space, whose distance metric is the generalized version of the
formula familiar from Cartesian geometry, SQRT(x2 + y2 + z2 +
\ldots{}).
\item distances are symmetric
\end{itemize}

\subsection{Issues}
\label{sec-22-2}
\begin{enumerate}
\item how do we determine the best \ref{sec-22-6}?
\begin{itemize}
\item important to minimize the number of dimensions
\begin{itemize}
\item for visualisation
\item and parsiomony of explanation
\begin{itemize}
\item as the number of free parameters increases there become
too few constraints on the possible configuration.
\end{itemize}
\end{itemize}
\item For a given dimensionality, we obtain two values: \hyperref[sec-22-3]{stress} and
         \ref{sec-22-4}
\item the number of objects should be at least 3-4 times greater
than the highest anticipated dimensionality. \texttt{samplaski}
paragraph 17
\end{itemize}
\item inherent underlying \hyperref[sec-22-7]{asymmetries}?
\begin{itemize}
\item \texttt{samplaski} paragraph 12
\item a number of models for dealing with this
\begin{itemize}
\item \ref{sec-22-10}
\end{itemize}
\end{itemize}
\item \ref{sec-22-8}
\end{enumerate}

\subsection{Stress}
\label{sec-22-3}
\begin{itemize}
\item \textbf{goodness of fit measure}
\item Discrepancies between the actual data values and the derived
underlying distances are accounted for in a goodness-of-fit
measure called "stress": as the number of dimensions increases,
stress decreases, and choosing between configurations of
different dimensionalities becomes an issue. (Samplaski)
\end{itemize}

\subsection{r2}
\label{sec-22-4}
\begin{itemize}
\item r-squared
\item percentage of the variability of the data being explained by the
solution
\end{itemize}

\subsection{elbows}
\label{sec-22-5}
\begin{itemize}
\item plot the stress and r2 values for several dimensionalities
\item look for "elbows" (inflection points) in the plots.
\item If an elbow exists, then the higher-dimensional solutions are
not giving significant additional explanation--the plot suddenly
flattens out.
\item \texttt{samplaski} paragraph 17
\end{itemize}

\subsection{dimensionality}
\label{sec-22-6}
\begin{itemize}
\item choose dimensionality on the bases of clarity and logical
interpretation.
\item one dimension above or below "optimal" as indicated by the
stress/r2 values might be better:
\begin{enumerate}
\item if there is a clear interpretation given an added dimension; or
\item if one configuration is easier to visualize (e.g 3-D vs 4-D
solution),
\begin{itemize}
\item especially in a situation where it is unclear what can be
gained in explanatory power by using the extra dimension.
\end{itemize}
\end{enumerate}
\end{itemize}

\subsection{asymmetry}
\label{sec-22-7}
-poor fit can be caused by several factors

\subsection{Exemplars}
\label{sec-22-8}
\begin{itemize}
\item In an MDS analysis of N objects, one of which is an exemplar,
the only way to minimize distortion (i.e., stress) is to place
the exemplar at the center of the configuration and arrange the
other objects around it.
\item \texttt{samplaski} note 15
\end{itemize}

\subsection{INDSCAL}
\label{sec-22-9}
\begin{itemize}
\item deals with idiosyncrasies of subjects ratings
\item takes one matrix per subject
\item \texttt{samplaski} paragraph 13
\end{itemize}

\subsection{ASCAL}
\label{sec-22-10}
\begin{itemize}
\item deals with possible asymmetries underlying the data
\item \texttt{samplaski} paragraph 14
\end{itemize}

\subsection{CA}
\label{sec-22-11}
\begin{itemize}
\item \textbf{cluster analysis}
\item (Tversky, 1977; Tversky and Gati, 1982; Tversky and
Hutchinson, 1986)
\item helps with problem of highly seperable dimensions
\item There is a family of CA models, but they all work similarly:
given a proximity or distance matrix, some method is used to
pick the pair of objects most like each other, group them into a
single cluster, and derive a new reduced matrix. When the
process is finished, the objects will be grouped into a binary
tree structure (exactly two branches descend from each node, and
the objects are "leaves" at the termini of the final branches),
where the distance between any pair of objects is related to the
length of the path along the branches separating them.
\end{itemize}

\subsection{PMDS}
\label{sec-22-12}
\begin{itemize}
\item \textbf{Probabalistic MDS}
\item \texttt{Samplaski} paragraph 23
\item assumed there is euclidean space
\item onjects are probability distributions
\item variance
\item PMDS is a technique still under development (2005)
\item \texttt{PROSCAL}
\end{itemize}
% Emacs 24.3.1 (Org mode 8.2.5g)
\end{document}
